\documentclass[11pt]{article}
\usepackage{float}
\usepackage{graphicx}
\usepackage{tabularx}
\usepackage{adjustbox}
\usepackage{amsmath,amssymb,trimclip,adjustbox}
%\usepackage[utf8]{inputenc}
%\usepackage[T1]{fontenc}
\usepackage{textcomp}
\usepackage{booktabs}
\newcommand{\Csh}{C\includegraphics{hash-symbol}}
\begin{document}
	\begin{titlepage}
		\begin{center}
			\Large{Warsaw University of Technology's}\\
			\Large{Faculty of Mathematics and Information Science}\\
			[0.3in]
			\begin{figure}[H]
				\centering
				\includegraphics[width=0.4\linewidth, height=0.25\textheight]{./media/uni_logo.jpeg}
				\label{Figure:f04}
			\end{figure}
			\Large{\bfseries Knowledge Representation and Reasoning}\\
			[0.3in]
			\Large{\bfseries Project number 2:}\\
			\Large{\bfseries Deterministic Action With Cost}\\
			\Large{\bfseries Supervisor: Dr Anna Radzikowska}\\
			[0.3in]
			\Large{\bfseries User Guide}\\
			[0.3in]
			\textsc{\Large{Created By}\\
				Rishabh Jain,
				Rahul Tomer,
				Kuldeep Shankar,\\ 
				Alaa Abboushi,
				Haran Dev Murugan,\\
				Bui Tuan Anh.\\}
		\end{center}	
	\end{titlepage}
	\tableofcontents
	\newpage
	\section{Introduction}\label{sec:intro}
	A dynamic system is designed for deterministic actions with cost. This document will help a user to understand the design and functionality of the system.
Following work flow is performed based on our Example 1 (Travel -Fuel – Reserve) from our drafted functional document. As system is dynamic hence other examples can be verified exactly same way as shown below.
Numerical Indices shown in images have been explained respectively followed by image.

System has been designed in C\# language.
	\section{Home Screen}\label{sec:homescreen}
	\bfseries{Legend:}
\begin{enumerate}
	\item Status \\ Displays the status of the project (If the fluents are submitted, if the actions are 	added, If the states are added etc.)\\
 	\item Scenario Build \\  This button is a drop-down menu button used for adding fluents and actions needed 	for scenario building. (Refer the Next image)\\
	\item Show Transitions\\ This Button opens the show states form that lets us Set the initial state and then shows the generated states and transitions of the system\\
	\item Queries\\ This Button opens the queries form window that lets us check a query against the 	scenario that we have built.
	\item	Clear Data\\ This Button clears all the data inputted by the user and reset the system.
\end{enumerate}



		 
\end{document}
